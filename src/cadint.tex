%%%%%%%%%%%%%%%%%%%%%%%%%%%%%%%%%%% CAD Interfaces %%%%%%%%%%%%%%%%%%%%%%%%%%%%%%%%%%%%%%%%%%%%%

\section{CAD Interfaces}

\subsection{The University of Wisconsin Unified Workflow}
The University of Wisconsin Unified Workflow, known as (UW)$^2$ is a mechanism by which 
monte carlo code agnostic metadata can be attached into DAGMC geometry. A DAGMC geometry
file is a MOAB database, which when on disk resides as a HDF5 formatted file. The C++
parts of the Python for Nuclear Engineering (PyNE) toolkit are used as a \textit{Lingua Franca}
to aid in the translation of metadata. Specifically, we use the Material, Tally, Particle 
and Name classes.
\subsection{Materials}
The \texttt{Material} class is the most fundamental class is used in UW$^2$, where we create and store
materials in their in memory (object) form, and also where the I/O methods are for the writing 
of the material object. The material object exists as a standard map of integer nuclide ID numbers
and their appropriate mass fractions, when required we have functions available to call which will
write their MC code specific versions, for example to Fluka or MCNP. 
\subsection{Tallies}
\subsection{\texttt{uwuw\_preproc}}
The general workflow of using UW$^2$ is to make a PyNE material library which contains all the materials
which are used in your problem. The \texttt{uwuw\_preproc} tool is then used to extract the material 
objects from the material library and inserts them into the DAGMC geometry file. This DAGMC geometry
file can now be run in any of the UW$^2$ enabled MC codes, but with the knowledge that the same original
material description is used in each code. 
\subsection{Limitations of UW$^2$}
It should be noted, that using the UW$^2$ workflow inputs the same material description, but the MC 
code may not be capable of describing the material in its original description, for example whilst
Fluka can describe any nuclide arbitrarily by defining its density, atomic and nucleon number it only
contains a limited selection for neutron cross sections below 20 MeV, with only the isotopes of 
hydrogen, helium, lithim, boron, iodine, xenon, caesium, uranium and some of plutonium descibed in
any isoptopic way. This means that physics aside, the ``low energy'' neutron material definitions are
inherently different, but this is order to match the requirements of the physics code and not a limitation 
of the workflow itself. Similarly to Fluka, Geant4 when suitable neutron transport cross sections are not 
available it will choose some defaults using the cross sections of nuclides with similar nucleon numbers
and atomic numbers.

\subsection{FluDAG}
The DAGMC implementation using Fluka is known as FluDAG. It uses the FluGG interface that exists
in FLuka. The function wrappers have distinct simple C style function arguments and store any 
C++ state behind this layer. The FluGG interface was originally written under the assumption
that unambiguous results to any of the geometric query functions can be given, such as 
given a particle location, determine the cell it is bounded by. With DAGMC we need to include some
additional state, stored in the RayHistory class which contains information regarding the last triangle
that was crossed, this state (if used) should be reset when we change direction, cross boundaries, and
when a history ends. There is some complexity regarding the internal program flow of Fluka, Specifically
some of the logic of electrons transport ``sensing steps'' and regarding when particles cross boundaries. 
Especially electrons crossing boundaries, which can logically cross a boundary but end up in the cell in 
which it began. The combination of DAGMC state and the need to know when a particle was taking a sensing step 
means that we needed to access a some of the internal Fluka common blocks which complicates some of the
software library design.

\subsection{DagSolid}
The DAGMC implementation for the Geant4 interface was originally written by. As part of this contract the 
DagSolid implementation was improved upon, unit tests added and a specific DagGeant4 executable was written
which allows the loading and running of the UW$^2$, reading material objects and tallies from the geometry 
and instanciates the appropriate Geant4 \texttt{G4Elements}, \texttt{G4Isotopes} and \texttt{G4Materials} and the 
appropriate \texttt{Detectors}. 

\subsection{HZETran}

\subsection{Acceptence Testing}
\subsection{Benchmarks}
\subsubsection{ATIC}



